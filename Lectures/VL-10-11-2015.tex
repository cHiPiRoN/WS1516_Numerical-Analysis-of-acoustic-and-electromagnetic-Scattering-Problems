\documentclass[../skript.tex]{subfiles}

	\begin{theorem}[Green's representation theorem]\label{c1se2thm11}
		\textbf{(i) } Any $y\in C^2(\Omega)\cap C^1(\overline{\Omega})$ satisfies for any $x\in\Omega$
		\[
			u(x) = \int_{\partial\Omega}\left[ \Phi(x,y)\frac{\partial u(y)}{\partial\nu} - u(y)\frac{\partial\Phi}{\partial\nu(y)} \right]\,dS(y) - \int_\Omega \Phi(x,y) (\kappa^2u(y)+\Delta u(y))\,dy
		\]
		(where $\nu$ denotes the outer normal of $\Omega$).\newline\newline\noindent
		\textbf{(ii) } Let $\Omega^c \coloneqq\mathbb{R}^3\setminus\Omega$ and let $u\in H^1_{loc}(\mathbb{R}^3)$ satisfy the Sommerfeld radiation condition, and solve
		\[
			\Delta u + \kappa^2 u = 0,\quad\text{in }\Omega^c.
		\]
		Then, the \emph{Green's formula} holds:
		\[
			u(x) = \int_{\partial\Omega}\left[ u(y)\frac{\partial\Phi(x,y)}{\partial\nu(y)} - \Phi(x,y)\frac{\partial}{\partial\nu}u(y) \right]\,dy,\quad x\in\mathbb{R}^3\setminus\overline{\Omega}.
		\]
		\textbf{(iii) } In the situation of $(ii)$, the far-field pattern of $u$ reads for $\hat{x} = \frac{x}{|x|}\in\mathcal{S}^2$
		\[
			u_\infty(\hat{x}) = \frac{1}{4\pi}\int_{\partial\Omega}\left[ u(y)\frac{\partial}{\partial\nu(y)}e^{-i\kappa\hat{x}\cdot y} - e^{-i\kappa\hat{x}\cdot y}\frac{\partial u}{\partial\nu}(y) \right]\,dS(y).
		\]
	\end{theorem}

	\begin{proof}
		\textbf{(i)} we know from \cref{c1se2thm7} (see addendum) that
		\[
			-u(x) = \int_\Omega u(y)(\kappa^2 + \Delta_y)\Phi(x,y)\,dy.
		\]
		The assertion follows from integrating by parts.\newline\newline\noindent

		\textbf{(ii)} Let $R>0$ and $K = B_R(0)$, with $\overline{\Omega}\subseteq K$.
		Define $D=K\setminus\overline{\Omega}$. Apply \textbf{(i)} to $D$ and obtain, for $x\in D$,
		\[
			u(x) = \int_{\partial\Omega} + \left[ \int_{\partial K}\right]\, \left[\Phi(x,y)\frac{\partial u}{\partial\nu}(y) - u(y)\frac{\partial}{\partial\nu(y)}\Phi(x,y)\right]\,dS(y).
		\]
		We want to show, that the $\int_{\partial K}$-integral vanishes for large $R\to \infty$. From the proof of \cref{c1se2thm4} we know, that $\int_{\partial K} |u|^2\,dS \leq O(1)$ as $R\to\infty$. From the Sommerfeld radiation condition of $u$ and $\Phi$, we obtain with the Cauchy-Schwarz inequality
		\begin{IEEEeqnarray*}{rCl}
			\int_{\partial K} ...\,dS &=& \int_{\partial K} \Phi(x,y)\left[ \frac{\partial u}{\partial\nu}(y) - i\kappa u(y) \right] + u(y) \left[ i\kappa\Phi(x,y) - \frac{\partial}{\partial\nu(y)}\Phi(x,y) \right]\,dS(y)\\
			&\overset{C.S.}\leq& \underbrace{\|\Phi\|_{L^2(\partial K, dy)} O\left(\frac{1}{r^2}\right) + \|u\|_{L^2(\partial K)}O\left(\frac{1}{r^2}\right)}_{\to 0,\text{ as }R\to\infty}.
		\end{IEEEeqnarray*}
		From \cref{c1se2pro6} we know ($\hat{x}=\frac{x}{|x|}$):
		\[
			\Phi(x,y) = \frac{1}{4\pi |x|} e^{i\kappa|x|} e^{-i\kappa\hat{x}\cdot y} + O\left(\frac{1}{|x|^2}\right).
		\]
		We plug this into the formula of \textbf{(ii)} and obtain for $|x|>>1$:
		\begin{IEEEeqnarray*}{rCl}
			u(x) &=& \frac{1}{4\pi} e^{i\kappa w}\int_{\partial\Omega}\left[ u(y)\frac{\partial}{\partial\nu(y)}e^{-i\kappa\hat{x}\cdot y} - e^{-i\kappa\hat{x}\cdot y}\frac{\partial}{\partial\nu}u(y) \right]\,dy + O\left(\frac{1}{|x|^2}\right).
		\end{IEEEeqnarray*}
		Uniqueness of the far-field pattern proves \textbf{(iii)}.
	\end{proof}


\section{Density results}\label{c1se3}

	In this section we collect some density results, useful in the labour study of inverse scattering. Recall: The refraction index $n$ satisfies $|n(x)| = 1$ for $|x| > \alpha$.
	\begin{lemma}\label{c1se3thm12}
		Let $\beta > \alpha$, $K\coloneqq B_\beta(0)$. Then there exist constants $M>0,C>0$, such that for any $z\in\mathbb{C}^3$ with $z\cdot z = 0$ and $|z|\geq M$ there exists a solution $u_z\in H^1(K)$ to the equation
		\[
			\Delta u_z + \kappa^2 n u_z = 0,\quad\text{in }K,
		\]
		of the form
		\[
			u_z(x) = e^{z\cdot x} (1+v_z(x)),\quad x\in K,
		\]
		and $v_z$ satisfies 
		\[
			\|v_z\|_{L^2(K)} \leq \frac{C}{|z|}.
		\]
	\end{lemma}

	\begin{proof}
		The technical lemma will be proven later.
	\end{proof}

	\begin{theorem}[1st density theorem]\label{c1se3thm13}
		Let $\Omega\subseteq\mathbb{R}^3$ be bounded and let $n_1,n_2\in L^\infty(\Omega)$ such that $(n_1-1)$ and $(n_2-1)$ have a compact support in $\Omega$. Then, the following linear hall of products is dense in $L^2$:
		\[
			X\coloneqq span\{u1u2:\,\forall j=1,2\,u_j\in H^1(\Omega)\text{ solves }\Delta u_j+\kappa^2 n_ju_j = 0\text{ in }\Omega\}
		\]
	\end{theorem}

	\begin{proof}
		Let $\beta > 0$ such that $\overline{\Omega}\subseteq B_\beta(0)$ and let $g\in L^2(\Omega)$ be $L^2(\Omega)$-Orthogonal to the set $X$. We have to show, that $g$ is zero. Then $X$ is dense.\newline\noindent
		Fix $y\in\mathbb{R}^3\setminus\{0\}$ and choose a unit vector $\hat{a}\in\mathbb{R}^3$ ($\hat{}$ means unit length) and some $b\in\mathbb{R}^3$with $|b|^2 = |y|^2 + s^2$ such that $\{y,\hat{x},b\}$ is an orthogonal system in $\mathbb{R}^3$ and some $s>0$ (real).\newline\noindent
		Define the following elements of $\mathbb{C}^3$:
		\begin{IEEEeqnarray*}{rCl}
			z^1&\coloneqq& \frac{1}{2} b-\frac{i}{2}(y+s\hat{a})\\
			z^2&\coloneqq& -\frac{1}{2}b-\frac{i}{2}(y-s\hat{a}).
		\end{IEEEeqnarray*}
		Then $z^1\cdot z^1 = 0 = z^2\cdot z^2$. We have 
		\[
			|z^1|^2 = |z^2|^2 = \frac{1}{4}\left[|b|^2 + |y|^2 + s^2\right]\geq\frac{1}{4}s^2.
		\]
		Furthermore $z^1 + z^2 = -iy$.
		From \cref{c1se3thm12} we obtain solutions to $\Delta u_j+\kappa^2 n_ju_j = 0$ in $B_\beta(0)$ with
		\[
			u_j(x) = e^{z^j\cdot x}\left[ 1+v_{z_j}(x) \right].
		\] 
		These are in particular solutions when restricted to the smaller domain $\Omega$. The orthogonal property of $g$ yields
		\begin{IEEEeqnarray*}{rCl}
			0 &=& \int_\Omega e^{(z^1+z^2)x}[1+v_{z_1}^2 + v_{z_2}^2 + v_{z_1}v_{z_2}]g\,dx.
		\end{IEEEeqnarray*}
		From \cref{c1se3thm12} we obtain 
		\[
			\|v_j\|_{L^2(\Omega)} \leq \frac{C}{|z_j|} \leq \frac{2C}{s}.
		\]
		Last inequality follows from a previous equation. From Cauchy's inequality and the limit $s\to 0$ that
		\[
			0 = \int_\Omega e^{-i y\cdot x}g(x)\,dx,\quad\forall y\in\mathbb{R}^3\setminus\{0\}.
		\]
		So, the Fourier-Transform of $g$ vanishes (it is defined like in our equation). Thus $g = 0$.
	\end{proof}

	The second density theorem is the following.

	\begin{theorem}[2nd density theorem]\label{c1se3thm14}
		Let $n\in L^\infty(\mathbb{R}^3)$ with $n(x) = 1$ for $|x|>\alpha$. Let $\beta>\alpha$ and define 
		\[
			H\coloneqq \left\{ v\in H^1(B_\beta(0))|\,\Delta v+\kappa^2nv=0\text{ in }B_\beta(0)\right\}
		\]
		the set of Helmholtz-solutions. Let for any $\hat{\theta}\in\mathcal{S}^2$, $u(\cdot,\hat{\theta})$ denote the total field corresponding to the incident plain wave $e^{i\kappa\hat{\theta}\cdot x}$ (plain wave from direction $\hat{\theta}$). Then, the span of all plain wave solutions,
		\[
			span\{u(\cdot,\hat{\theta}):\,\hat{\theta}\in\mathcal{S}^2\},
		\] 
		is dense
		\[
			H|_{B_\alpha(0)}\text{ in }L^2(B_\alpha(0))\text{ - norm}.
		\]
	\end{theorem}

	\begin{proof}
		Let $K=B_\alpha(0)$ and $\langle v,w\rangle$ the $L^2$ product over $K$, so
		\[
			\langle v,w\rangle \coloneqq \int_K v\overline{w}\,dx.
		\]
		Let $v$ be in the closure of $H$ with the following property:
		\[
			\langle v,u(\cdot,\hat{\theta})\rangle = 0
		\]
		for all $\hat{\theta}\in\mathcal{S}^2$. Recall the operator 
		\[
			T:v\mapsto \kappa^2\int_{|x|<\alpha} (1-n(y))\Phi(x,y)w(y)\,dy
		\]
		and the Lippmann-Schwinger equation
		\[
			u(\cdot,\theta) = (1+T)^{-1}u^{in}(\cdot,\theta).
		\]
		With the adjoint operator $T^*$, that satisfies
		\[
			\langle T^*v,w\rangle = \langle v,Tw\rangle,\quad\forall v,w\in X,
		\]
		and the fact that
		\[
			[[1+T]^{-1}]^* = (1+T^*)^{-1},
		\]
		we obtain that
		\begin{IEEEeqnarray*}{rCl}
			0 &=& \langle v,(1+T)^{-1}u^{in}(\cdot,\hat{\theta})\\
			&=& \langle(1+T^*)^{-1}v,u^{in}(\cdot,\hat{\theta}),
		\end{IEEEeqnarray*}
		for all $\hat{\theta}\in\mathcal{S}^2$. We compute $T^*$:\newline\noindent
		Since, for all $w_1,w_1\in L^2(K)$ we have
		\begin{IEEEeqnarray*}{rCl}
			\frac{1}{\kappa^2}\langle T^*w_1,w_2\rangle &=& \frac{1}{\kappa^2}\langle w_1,Tw_2\rangle\\
			&=& \int_K \int_K w_1(x) \overline{(1-n(y))}\overline{\Phi(x,y)}\overline{w_2(y)}\,dy\,dx\\
			&=& \langle (1-\overline{n})\int_K w_1(x)\overline{\Phi(x,\cdot)}\,dx, w_2\rangle,
		\end{IEEEeqnarray*}
		we have
		\[
			T^*w_1 = \kappa^2\overline{(1-n)}\int_K w_1(y)\overline{\Phi(\cdot,y)}\,dy.
		\]
		Thus $w\coloneqq (1+T^*)^{-1}v$ satisfies
		\[
			v(x) = w(x) + \kappa^2 (1-\overline{n(x)})\int_K\overline{\Phi(x,y)}w(y)\,dy.
		\]
		We now set 
		\[
			\tilde{w}(x) \coloneqq \int_K \overline{w(y)} \Phi(x,y)\,dy,\quad x\in\mathbb{R}^3.
		\]
		Then, $\tilde{w}$ is a volume potential for $\bar{w}$ in the sense of \cref{c1se2thm7}: $w\in H^1_{loc}(\mathbb{R}^3)$ and
		\[
			\int_{\mathbb{R}^3}\left( \nabla\tilde{w}\cdot\nabla\overline{\Psi}-\kappa^2\tilde{w}\overline{\Psi} \right)\,dx = \int_K \overline{w}\overline{\Psi}\,dx,\quad\forall \Psi\in H^1(\mathbb{R}^3)\text{ with compact support}.
		\]
		The far-field pattern $\tilde{w}_\infty$ vanishes. Indeed, as \cref{c1se2thm11} (iii) and the divergence theorem imply for any $\hat{\Theta}\in\mathcal{S}^2$:
		\[
			\overline{\tilde{w}_\infty(\hat{\theta})} = \int_K w(y)e^{i\kappa\hat{\theta}\cdot y}\,dy = 0,
		\]
		since $w$ is orthogonal to the plain waves. Rellichs theorem implies then, that $\tilde{w} = 0$ outside of $K$! Consider a sequence $v_j\in H$ with $v_j\to v$ in $L^2(H)$ as $j\to\infty$. From
		\[
			v = w + \overline{\kappa^2(1-n)\tilde{w}}
		\]
		we get 
		\begin{equation}\label{c1se3thm14eqn*}\tag{*}
			\int_K \overline{v}v_j\,dx = \int_K \overline{w}v_j\,dx + \kappa^2\int_K(1-n)\tilde{w}v_j\,dx.
		\end{equation}
		Since $\tilde{w}$ vanishes outside $K$, we obtain from $v_j\in H$ 
		\[
			\int_{|x|<\beta} (\nabla v_j\cdot\nabla\tilde{w} - \kappa^2 v_j\cdot\tilde{w})\,dx = -\kappa^2\int_K(1-n)v_j\tilde{w}\,dx.
		\]
		We extend $v_j$ to a $H^1(\mathbb{R}^3)$ function with compact support. The property of $\tilde{w}$ implies
		\[
			\int_{|x|<\beta}(\nabla\tilde{w}\cdot\nabla v_j - \kappa^2\tilde{w}\cdot v_j)\,dx = \int_K (1-n)v_j\tilde{w}\,dx.
		\]
		The previous formulas imply
		\[
			-\kappa^2\int_K (1-n)v_j\tilde{w}\,dx = \int_K\overline{w}v_j\,dx.
		\]
		This and \cref{c1se3thm14eqn*} give
		\[
			\int_K\overline{v}v_j = 0,\quad\forall j\in\mathbb{N}.
		\]
		In the limit $j\to\infty$ we obtain
		\[
			\|v\|_{L^2(K)} = 0 \Longleftrightarrow v = 0.
		\]
		This proves density.
	\end{proof}
