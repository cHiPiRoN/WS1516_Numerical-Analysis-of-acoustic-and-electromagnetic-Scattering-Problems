\documentclass[../skript.tex]{subfiles}

% Chapter 2
\chapter{Discretization of the direct scattering problem}\label{c2}

\section{Dealing with infinite domains}\label{c2se1}
	
	Recall the direct scattering \cref{c1se2prbS} from \cref{c1}: \newline\noindent
	Given some incident wave $u^{in}(x)$ of plain-wave type, i.e. 
	\[
		u^{in}(x) = c e^{i\kappa\hat\theta\cdot x},
	\]
	with direction $\hat\theta\in\mathcal{S}^{d-1}$ (unit sphere in $\mathbb{R}^d$) and (real) frequency $\kappa>0$, find the total field $u\in H^1_{loc}(\mathbb{R}^d)$ such that for all $\Psi\in H^1(\mathbb{R}^d)$ with compact support it holds that
	\[
		\int_{\mathbb{R}^d} (\nabla u\cdot\nabla\overline{\Psi}-\kappa^2nu\overline{\Psi})\,dx = 0,
	\]
	where $n\in L^\infty(\Omega)$ function with $\Re{n},\Im{n}\geq 0$.
	Additionaly the scattered field $u^s\coloneqq u-u^{in}$ should solve the Sommerfeld radiation condition
	\[
		\frac{\partial u^s(x)}{\partial r} - i\kappa u^s(x) = O\left(\frac{1-d}{2}\right),
	\]
	for $|x|=r\to\infty$ uniformly in $\hat{x}=\frac{x}{|x|}\in\mathcal{S}^{d-1}$.\newline\newline\noindent

	The scattering problem is an infinite-dimensional variational problem on an infinite (unbounded) domain.\newline\newline\noindent
	\textbf{Aim of this chapter: } Approximate \cref{c1se2prbS} in a finite computational complexity and time.\newline\noindent
	This section deals with the restriction / truncation to bounded domains. \newline\newline\noindent
	\textbf{Common approach: } Simply replace $\mathbb{R}^d$ by some bounded domain $\Omega$ and put some artificial absorbing boundary condition on $\partial\Omega$. The principle is to allow all waves to leave the domain without any artifical scattering (i.e. the initial wave, that may be scattered by some objects in $\Omega$, should be able to leave the domain without any harm). At the boundary of the domain, nothing that's leaving should be scattered inside again (as $\Omega$ is just used for being able to compute something).\newline\newline\noindent
	\textbf{Reference: } Lecture-notes of Olof Runborg, KTH Stockholm.

	\subsection{One-dimensional illustration}
		In 1d (and only in 1d), it is possible to design exact local absorbing boundary conditions. W.l.o.g. let the artificial domain $\Omega = B_L(0)$ be clutered and let $\overline{B_\alpha(0)}\subseteq B_L(0)$ ($L>\alpha$, where $\alpha$ is the parameter from \cref{c1}, $n|_{B_\alpha(0)^c}=1$). Close to $\partial\Omega = \{-L,L\}$, the medium is homogeneous and hence the scattered wave $u^s\coloneqq u-u^{in}$ satisfies the Helmholtz-equation
		\[
			u_{xx}^s - \kappa^2 u^s = 0,
		\] 
		for $|x\pm L|$ sufficiently small.

		Therefore 
		\begin{equation}\label{c2se1eqn1}
			u^s(x) = \underbrace{c_0e^{-i\kappa x}}_{A} + \underbrace{c_1e^{i\kappa x}}_{B},
		\end{equation}
		with $c_0,c_1\in\mathbb{R}$ and $x$ close to $\pm L$. The part $A$ of the solutions travels to the left of the interval, the part $B$ travels to the right of the interval. As stated above: If we were sitting on the boundary, we would like to be able to allow waves to pass the boundary from the inside. If a wave comes from the outside, we don't let it pass. Therefore we want to design the absorbing boundary condition at $-L$ such that it accepts any $c_0$, but prevents $c_1\not=0$. This is archieved by applying the operator $\frac{\partial}{\partial x}+i\kappa$. Why this? Because:
		\begin{IEEEeqnarray*}{rCl}
			\left(\frac{\partial}{\partial x}+i\kappa\right) u^s &=& -c_0i\kappa e^{-i\kappa x} + c_1i\kappa e^{i\kappa x} + c_0i\kappa e^{-i\kappa x}+c_1i\kappa e^{i\kappa x}\\
			&=& 2 c_1 i \kappa e^{i\kappa x},
		\end{IEEEeqnarray*}
		and this shows that $\left(\frac{\partial}{\partial x}+i\kappa\right)$ doesn't `see' waves leaving $\Omega$ (i.e. the $c_0$ part of the solution), and incoming waves can be avoided by the boundary condition
		\[
			\left(\frac{\partial}{\partial x}u^s + i\kappa u^s\right)(-L) = 0.
		\]
		Similarily, at $x=+L$ we impose
		\[
			\left(\frac{\partial}{\partial x} u^s - i\kappa u^s\right)(+L) = 0
		\]
		(calculation shows this again). Introducing the concept of an outer normal $\nu$, i.e. $\nu(\pm L) = \pm 1$, we can rewrite both conditions  as
		\[
			\frac{\partial u^s}{\partial \nu}(x) - i\kappa u^s(x) = 0,
		\]
		for $x\in \partial\Omega = \{-L,L\}$. This leads to the following inhomogeneous boundary condition for the total wave $u=u^s + u^{in}$:
		\begin{equation}\label{c2se1eqn2}
			\left(\frac{\partial u}{\partial\nu}-i\kappa u\right)(x) = \left(\frac{\partial u^{in}}{\partial\nu} - i\kappa u^{in}\right)(x),\quad x\in\partial\Omega = \{-L,L\}.
		\end{equation}
		\begin{example}
			$u^{in} = e^{-i\kappa x}$ is a wave travelling from right to left. Then the absorbing boundary condition reads
			\begin{IEEEeqnarray*}{rCl}
				\left(\frac{\partial u}{\partial\nu}-i\kappa u\right)(-L) &=& 0,\\
				\left(\frac{\partial u}{\partial\nu}-i\kappa u\right)(L) &=& 2i\kappa u^{in}(L).
			\end{IEEEeqnarray*}
		\end{example}

		We shall emphasize that the scattering problem on $\Omega$ with absorbing boundary condition \cref{c2se1eqn2},
		\begin{IEEEeqnarray*}{rCl}
			\Delta u + i\kappa n u &=& 0,\quad\text{in }\Omega\\
			\frac{\partial u}{\partial\nu} - i\kappa u &=& \frac{\partial u^{in}}{\partial \nu} - i\kappa u^{in},\quad\text{on }\partial\Omega,
		\end{IEEEeqnarray*}
		reproduces the solution of \cref{c1se2prbS} in $\Omega$.\newline\newline\noindent

	\subsection{Multi-dimensional case}

		In higher dimensions, there is no simple form like \cref{c1se2eqn1} for this scattered wave close to the artificial boundary $\partial\Omega$. 
		Exact absorbing boundary conditions will be necesarrily non-local and computationally \underline{demanding}.\newline\newline\noindent
		A cheaper and popular approach is the following:\newline\noindent
		Assume that the scattered wave in the full problem is close to a spherical wave in a vicinity of the artificial boundary. If we choose $\Omega = B_L(0)$, then the 1d-arguments can be applied. If $u^s(x) \approx e^{i\kappa|x|}$ (i.e. the wave is approximately spherical; spherical waves have no angular dependence) close to $\partial\Omega$, then $\frac{\partial u^s}{\partial\nu}-i\kappa u^s \approx i\kappa e^{ik|x|}-i\kappa e^{i\kappa |x|} = 0$  on $\partial\Omega$.\newline\noindent
		This motivates the simplest absorbing bc in 2 and 3 dimensions:
		\begin{equation}\label{c2se1eqn3}
			\frac{\partial u}{\partial\nu}-i\kappa u = \frac{\partial u^{in}}{\partial\nu} - i\kappa u^{in},\quad\text{on }\partial\Omega.
		\end{equation}
		We have seen in \cref{c1se2thm10} that $u^s$ is actually spherical, up to some perturbation of order $O\left(\frac{1}{|x|^2}\right)$:
		\[
			u^s(x) = \frac{e^{i\kappa|x|}}{|x|} u_\infty(\hat{x})+O\left(\frac{1}{|x|^2}\right),\quad\text{as }|x|\to\infty,
		\]
		i.e. `the scattered wave is spherical if we are far enough away from all obstacles'. \newline\noindent
		Hence, for sufficiently large $L$ ($\Omega=B_L(0)$), is absorbing boundary condition (\ref{c2se1eqn3}) appears reasonable.\newline\noindent
		For the remaining part of this chapter, our model problem reads
		\begin{problem}\label{c2se1prbSOmega}
			\begin{IEEEeqnarray*}{rCl}
				\Delta u + \kappa^2 nu &=& 0,\text{ in }\Omega\\
				\frac{\partial u}{\partial\nu}-i\kappa u &=& \frac{\partial u^{in}}{\partial\nu}-i\kappa u^{in},text{ on }\partial\Omega.
			\end{IEEEeqnarray*}
		\end{problem}
		\begin{remark}
			The absorbing boundary condition (\ref{c2se1eqn3}) can be seen as applying the Sommerfeld radiation condition at $\partial B_L(0)$ instead of $\infty$. \newline\noindent
			Note, that the choice of the boundary condition we made was just the easiest one. There are many alternatives. The most popular approach is the so called \emph{Perfectly Matched Layer} (PML). In this approach no bc on $\partial\Omega$ are imposed. The idea is to put around $\Omega$ some thin layer of artificial material (by changing $n$ / making it imaginary). This can also be used to absorb outgoing waves.
		\end{remark}








